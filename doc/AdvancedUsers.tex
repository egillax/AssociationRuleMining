% Options for packages loaded elsewhere
\PassOptionsToPackage{unicode}{hyperref}
\PassOptionsToPackage{hyphens}{url}
%
\documentclass[
]{article}
\usepackage{lmodern}
\usepackage{amssymb,amsmath}
\usepackage{ifxetex,ifluatex}
\ifnum 0\ifxetex 1\fi\ifluatex 1\fi=0 % if pdftex
  \usepackage[T1]{fontenc}
  \usepackage[utf8]{inputenc}
  \usepackage{textcomp} % provide euro and other symbols
\else % if luatex or xetex
  \usepackage{unicode-math}
  \defaultfontfeatures{Scale=MatchLowercase}
  \defaultfontfeatures[\rmfamily]{Ligatures=TeX,Scale=1}
\fi
% Use upquote if available, for straight quotes in verbatim environments
\IfFileExists{upquote.sty}{\usepackage{upquote}}{}
\IfFileExists{microtype.sty}{% use microtype if available
  \usepackage[]{microtype}
  \UseMicrotypeSet[protrusion]{basicmath} % disable protrusion for tt fonts
}{}
\makeatletter
\@ifundefined{KOMAClassName}{% if non-KOMA class
  \IfFileExists{parskip.sty}{%
    \usepackage{parskip}
  }{% else
    \setlength{\parindent}{0pt}
    \setlength{\parskip}{6pt plus 2pt minus 1pt}}
}{% if KOMA class
  \KOMAoptions{parskip=half}}
\makeatother
\usepackage{xcolor}
\IfFileExists{xurl.sty}{\usepackage{xurl}}{} % add URL line breaks if available
\IfFileExists{bookmark.sty}{\usepackage{bookmark}}{\usepackage{hyperref}}
\hypersetup{
  pdftitle={An introduction to the AssociationRuleMining package for advanced users.},
  hidelinks,
  pdfcreator={LaTeX via pandoc}}
\urlstyle{same} % disable monospaced font for URLs
\usepackage[margin=1in]{geometry}
\usepackage{color}
\usepackage{fancyvrb}
\newcommand{\VerbBar}{|}
\newcommand{\VERB}{\Verb[commandchars=\\\{\}]}
\DefineVerbatimEnvironment{Highlighting}{Verbatim}{commandchars=\\\{\}}
% Add ',fontsize=\small' for more characters per line
\usepackage{framed}
\definecolor{shadecolor}{RGB}{248,248,248}
\newenvironment{Shaded}{\begin{snugshade}}{\end{snugshade}}
\newcommand{\AlertTok}[1]{\textcolor[rgb]{0.94,0.16,0.16}{#1}}
\newcommand{\AnnotationTok}[1]{\textcolor[rgb]{0.56,0.35,0.01}{\textbf{\textit{#1}}}}
\newcommand{\AttributeTok}[1]{\textcolor[rgb]{0.77,0.63,0.00}{#1}}
\newcommand{\BaseNTok}[1]{\textcolor[rgb]{0.00,0.00,0.81}{#1}}
\newcommand{\BuiltInTok}[1]{#1}
\newcommand{\CharTok}[1]{\textcolor[rgb]{0.31,0.60,0.02}{#1}}
\newcommand{\CommentTok}[1]{\textcolor[rgb]{0.56,0.35,0.01}{\textit{#1}}}
\newcommand{\CommentVarTok}[1]{\textcolor[rgb]{0.56,0.35,0.01}{\textbf{\textit{#1}}}}
\newcommand{\ConstantTok}[1]{\textcolor[rgb]{0.00,0.00,0.00}{#1}}
\newcommand{\ControlFlowTok}[1]{\textcolor[rgb]{0.13,0.29,0.53}{\textbf{#1}}}
\newcommand{\DataTypeTok}[1]{\textcolor[rgb]{0.13,0.29,0.53}{#1}}
\newcommand{\DecValTok}[1]{\textcolor[rgb]{0.00,0.00,0.81}{#1}}
\newcommand{\DocumentationTok}[1]{\textcolor[rgb]{0.56,0.35,0.01}{\textbf{\textit{#1}}}}
\newcommand{\ErrorTok}[1]{\textcolor[rgb]{0.64,0.00,0.00}{\textbf{#1}}}
\newcommand{\ExtensionTok}[1]{#1}
\newcommand{\FloatTok}[1]{\textcolor[rgb]{0.00,0.00,0.81}{#1}}
\newcommand{\FunctionTok}[1]{\textcolor[rgb]{0.00,0.00,0.00}{#1}}
\newcommand{\ImportTok}[1]{#1}
\newcommand{\InformationTok}[1]{\textcolor[rgb]{0.56,0.35,0.01}{\textbf{\textit{#1}}}}
\newcommand{\KeywordTok}[1]{\textcolor[rgb]{0.13,0.29,0.53}{\textbf{#1}}}
\newcommand{\NormalTok}[1]{#1}
\newcommand{\OperatorTok}[1]{\textcolor[rgb]{0.81,0.36,0.00}{\textbf{#1}}}
\newcommand{\OtherTok}[1]{\textcolor[rgb]{0.56,0.35,0.01}{#1}}
\newcommand{\PreprocessorTok}[1]{\textcolor[rgb]{0.56,0.35,0.01}{\textit{#1}}}
\newcommand{\RegionMarkerTok}[1]{#1}
\newcommand{\SpecialCharTok}[1]{\textcolor[rgb]{0.00,0.00,0.00}{#1}}
\newcommand{\SpecialStringTok}[1]{\textcolor[rgb]{0.31,0.60,0.02}{#1}}
\newcommand{\StringTok}[1]{\textcolor[rgb]{0.31,0.60,0.02}{#1}}
\newcommand{\VariableTok}[1]{\textcolor[rgb]{0.00,0.00,0.00}{#1}}
\newcommand{\VerbatimStringTok}[1]{\textcolor[rgb]{0.31,0.60,0.02}{#1}}
\newcommand{\WarningTok}[1]{\textcolor[rgb]{0.56,0.35,0.01}{\textbf{\textit{#1}}}}
\usepackage{graphicx,grffile}
\makeatletter
\def\maxwidth{\ifdim\Gin@nat@width>\linewidth\linewidth\else\Gin@nat@width\fi}
\def\maxheight{\ifdim\Gin@nat@height>\textheight\textheight\else\Gin@nat@height\fi}
\makeatother
% Scale images if necessary, so that they will not overflow the page
% margins by default, and it is still possible to overwrite the defaults
% using explicit options in \includegraphics[width, height, ...]{}
\setkeys{Gin}{width=\maxwidth,height=\maxheight,keepaspectratio}
% Set default figure placement to htbp
\makeatletter
\def\fps@figure{htbp}
\makeatother
\setlength{\emergencystretch}{3em} % prevent overfull lines
\providecommand{\tightlist}{%
  \setlength{\itemsep}{0pt}\setlength{\parskip}{0pt}}
\setcounter{secnumdepth}{-\maxdimen} % remove section numbering

\title{An introduction to the AssociationRuleMining package for advanced users.}
\author{}
\date{\vspace{-2.5em}}

\begin{document}
\maketitle

\hypertarget{overview}{%
\section{Overview}\label{overview}}

To quickly summarise how to implement the AssociationRuleMining package:

\begin{enumerate}
\def\labelenumi{\arabic{enumi})}
\tightlist
\item
  The package implements two analysis frameworks, association rule and
  frequent pattern mining.
\end{enumerate}

Assuming the user has instantiated an appropriate and relevant cohort,
and extracted the relevant information using \texttt{FeatureExtraction}.

\begin{enumerate}
\def\labelenumi{\arabic{enumi})}
\setcounter{enumi}{1}
\tightlist
\item
  Each analysis framework of the package, has its own set of functions.
\end{enumerate}

\begin{itemize}
\item
  Association rule mining can be implemented using just two functions:
  \texttt{getInputFileForAssociationRules()} and
  \texttt{runAssociationRules()}.
\item
  Frequent pattern mining can be implemented using just two functions:
  \texttt{getInputFileForFrequentPatterns()} and
  \texttt{runFrequentPatterns()}. Additionally, the
  \texttt{getIdDataFrame()} provides a data frame object that indicates
  the presence or not of a sequence.
\end{itemize}

\textbf{Notes:}\\
- Input files should be .txt files.\\
- None of the functions are exported at the moment. Make sure to use
\texttt{devtools:load\_all()} after loading the package.\\
- For quick examples have a look at the
\protect\hyperlink{preparing-the-data-for-mining}{Preparing the data for
mining} and \protect\hyperlink{mining}{Mining} sections below.

\hypertarget{introduction}{%
\section{Introduction}\label{introduction}}

This vignette is designed in a way that briefly introduces the advanced
user to the AssociationRuleMining R package. By advanced user, we refer
to someone already familiar to the OHDSI framework as a start. The
package is designed to provide a pipeline for Association Rule Mining
(ARM) and Frequent Pattern Mining (FPM) against the OMOP-CDM, even
though with an acceptable data input format it can be used to generate
results for any general problem or use case.

The package makes use of the open source SPMF library, developed and
maintained by Philippe Fournier-Viger. SPMF is a Data Mining Java
library implementing a large collection of algorithms related to ARM and
FPM, as well as, clustering algorithms and time series mining. The
reader is encouraged to have a look at the
\href{http://www.philippe-fournier-viger.com/spmf/}{website of SPMF} and
explore the documentation and the vast list of implementations offered
by the library.

Finally, we would like to encourage the user to experiment with the
package and provide feedback, either through the
\href{https://github.com/mi-erasmusmc/AssociationRuleMining/issues}{GitHub
issue tracker} or by email. Have in mind that the package is in the
development phase and some features may break. Also, documentation about
the conceptual frameworks of ARM and FPM, as well as, documentation
related to the algorithms are in production. Interested users may want
to have a regular look at the
\href{https://github.com/mi-erasmusmc/AssociationRuleMining}{github}
site of the package for updates. Happy mining!

\hypertarget{implementation}{%
\section{Implementation}\label{implementation}}

\hypertarget{data-preparation}{%
\subsection{Data Preparation}\label{data-preparation}}

\hypertarget{getting-the-necessary-data-out-of-the-cdm}{%
\subsubsection{Getting the necessary data out of the
CDM}\label{getting-the-necessary-data-out-of-the-cdm}}

We assume that an appropriate cohort has been generated for which we
would like to extract relevant covariates (e.g.~first instance of
Myocardial Infarction) and is living on an SQL table. The cohort needs
to have the three least usual columns of subject\_id,
cohort\_start\_date, and cohort\_definition\_id. The FeatureExtraction
package can then be used to generate temporal covariates of interest for
the cohort. For detailed instructions of how to do that have a look at
the FeatureExtraction package.

A simple pipeline for generating covariates to be used for ARM is the
following:

\begin{Shaded}
\begin{Highlighting}[]
\CommentTok{#### Feature Extraction ####}
\NormalTok{covariateSettings <-}\StringTok{ }\KeywordTok{createCovariateSettings}\NormalTok{(}\DataTypeTok{useConditionOccurrenceAnyTimePrior =} \OtherTok{TRUE}\NormalTok{)}

\NormalTok{covariateData <-}\StringTok{ }\KeywordTok{getDbCovariateData}\NormalTok{(}\DataTypeTok{connection =}\NormalTok{ connection, }
                                    \DataTypeTok{cdmDatabaseSchema =}\NormalTok{ cdmdatabaseschema, }\CommentTok{#The database schema where the cdm lives}
                                    \DataTypeTok{cohortDatabaseSchema =}\NormalTok{ resultsDatabaseSchema, }\CommentTok{#The database schema where the cohort table lives}
                                    \DataTypeTok{cohortTable =} \StringTok{"diagnoses"}\NormalTok{, }\CommentTok{#Name of the cohort table}
                                    \DataTypeTok{rowIdField =} \StringTok{"subject_id"}\NormalTok{, }
                                    \DataTypeTok{covariateSettings =}\NormalTok{ covariateSettings, }
                                    \DataTypeTok{cohortTableIsTemp =} \OtherTok{TRUE}\NormalTok{) }\CommentTok{#If the cohort table is temporary or not}
\end{Highlighting}
\end{Shaded}

A simple pipeline for generating temporal covariates to be used for FPM
is the following:

\begin{Shaded}
\begin{Highlighting}[]
\NormalTok{TemporalcovariateSettings <-}\StringTok{ }\KeywordTok{createTemporalCovariateSettings}\NormalTok{(}\DataTypeTok{useConditionOccurrence =} \OtherTok{TRUE}\NormalTok{,}
                                                      \DataTypeTok{temporalStartDays =} \KeywordTok{seq}\NormalTok{(}\OperatorTok{-}\NormalTok{(}\DecValTok{60}\OperatorTok{*}\DecValTok{365}\NormalTok{), }\DecValTok{-1}\NormalTok{, }\DataTypeTok{by =} \DecValTok{1}\NormalTok{) ,}
                                                      \DataTypeTok{temporalEndDays =} \KeywordTok{seq}\NormalTok{(}\OperatorTok{-}\NormalTok{(}\DecValTok{60}\OperatorTok{*}\DecValTok{365}\NormalTok{)}\OperatorTok{+}\DecValTok{1}\NormalTok{, }\DecValTok{0}\NormalTok{, }\DataTypeTok{by =} \DecValTok{1}\NormalTok{))}

\CommentTok{# Extract covariates}
\NormalTok{TemporalcovariateData<-}\StringTok{ }\KeywordTok{getDbCovariateData}\NormalTok{(}\DataTypeTok{connection =}\NormalTok{ connection, }
                                           \DataTypeTok{cdmDatabaseSchema =}\NormalTok{ cdmdatabaseschema, }\CommentTok{#The database schema where the cdm lives}
                                           \DataTypeTok{cohortDatabaseSchema =}\NormalTok{ resultsdatabaseschema, }\CommentTok{#The database schema where the cohort table lives}
                                           \DataTypeTok{cohortTable =}\NormalTok{ cohorttable, }\CommentTok{#Name of the cohort table}
                                           \DataTypeTok{rowIdField =} \StringTok{"subject_id"}\NormalTok{, }
                                           \DataTypeTok{covariateSettings =}\NormalTok{ TemporalcovariateSettings, }
                                           \DataTypeTok{cohortTableIsTemp =} \OtherTok{TRUE}\NormalTok{) }\CommentTok{#If the cohort table is temporary or not}
\end{Highlighting}
\end{Shaded}

\hypertarget{preparing-the-data-for-mining}{%
\subsubsection{Preparing the data for
mining}\label{preparing-the-data-for-mining}}

To generate the appropriate input file for ARM, one should make use of
the function \texttt{getInputFileForAssociationRules()}. The analogous
function for FPM is \texttt{getInputFileForFrequentPatterns()}. These
functions generate the necessary input files that fulfill the adequate
input structure for the algorithms to be implemented. A word of caveat
is that the input format has to be a .txt file. Have a look at the SPMF
documentation for more details and acceptable alternatives. However,
AssociationRuleMining only supports .txt files as inputs at the moment.
As these functions generate a text file that is to be processed by the
algorithms, they do not need to be assigned to an R object.

\begin{Shaded}
\begin{Highlighting}[]
\KeywordTok{getInputFileForAssociationRules}\NormalTok{(}\DataTypeTok{covariateDataObject =}\NormalTok{ covariateData, }\DataTypeTok{fileToSave =} \StringTok{"AssociationRulesExample.txt"}\NormalTok{)}
\end{Highlighting}
\end{Shaded}

Note that the \texttt{covariateDataObject} argument takes the covariate
data object generated by \texttt{FeatureExtarction}. The
\texttt{fileToSave} argument takes the path of the file where the input
data is going to be stored and should end in `.txt'.

Similarly for FPM, the same notes as above hold.

\begin{Shaded}
\begin{Highlighting}[]
\KeywordTok{getInputFileForFrequentPatterns}\NormalTok{(}\DataTypeTok{covariateDataObject =}\NormalTok{ TemporalcovariateData, }\DataTypeTok{fileToSave =} \StringTok{"FrequentPatternsExample.txt"}\NormalTok{)}
\end{Highlighting}
\end{Shaded}

\hypertarget{mining}{%
\subsubsection{Mining}\label{mining}}

We would like to emphasize that there is no documentation yet that
describes the arguments that are accepted as inputs to the functions
calls. We try to describe them in some detail here.

\hypertarget{association-rule-mining}{%
\paragraph{Association Rule Mining}\label{association-rule-mining}}

To run an analysis of ARM, one needs to call
\texttt{runAssociationRules()}. The function requires four arguments to
be specified:

\begin{itemize}
\tightlist
\item
  \texttt{algorithm}: Which algorithm to run, currently only accepting
  one of ``Apriori'', ``Eclat'', ``FP-Growth'', ``Relim'' and should be
  quoted.\\
\item
  \texttt{inputFile}: Location and name of the file generated by
  \texttt{getInputFileForAssociationRules()}.\\
\item
  \texttt{outputFile}: Location and name of the file where the results
  should be saved. \textbf{Should be a .txt file}.\\
\item
  \texttt{minSup}: Minimum support for mined items.
\end{itemize}

Other arguments are possible to be specified, such as, maxLength,
indicating the maximum number of items mined in an itemset. However,
this is not applicable to every algorithm and it is not implemented at
the moment in this package.

An example can be the following:

\begin{Shaded}
\begin{Highlighting}[]
\NormalTok{associationSets <-}\StringTok{ }\KeywordTok{runAssociationRules}\NormalTok{(}\DataTypeTok{algorithm =} \StringTok{"Apriori"}\NormalTok{, }
                                       \DataTypeTok{inputFile =} \StringTok{"AssociationRulesExample.txt"}\NormalTok{, }
                                       \DataTypeTok{outputFile =} \StringTok{"AssociationRulesExample_Results.txt"}\NormalTok{,}
                                       \DataTypeTok{minsup =} \FloatTok{0.5}\NormalTok{ )}
\end{Highlighting}
\end{Shaded}

The function generates another text file saved as
``AssociationRulesExample\_Results.txt'' in the current working
directory. It prints on the screen the number of mined itemsets, the
time required, memory used and the location of where the output is
stored. It also, transforms the output to an R object and therefore
needs to be saved.

\hypertarget{frequent-pattern-mining}{%
\subsubsection{Frequent Pattern Mining}\label{frequent-pattern-mining}}

Similarly for FPM, one needs to call \texttt{runFrequentPatterns()}. The
function requires 4 mandatory arguments to be specified, and other
arguments related to each algorithm.

\begin{itemize}
\tightlist
\item
  \texttt{algorithm}: Which algorithm to run, currently only supporting
  one of ``SPAM'', ``SPADE'', ``prefixSpan'' and should be quoted.\\
\item
  \texttt{inputFile}: Location and name of the file generated by
  \texttt{getInputFileForFrequentPatterns()}.\\
\item
  \texttt{outputFile}: Location and name of the file where the results
  should be saved. \textbf{Should be a .txt file}.\\
\item
  \texttt{minsup}: Mininum support for mined sequences.
\end{itemize}

Additionally, some of the algorithms accept several additional
arguments:

\begin{itemize}
\tightlist
\item
  \texttt{minLength}: The minimum length required for a sequence,
  defaults to 1.\\
\item
  \texttt{maxLength}: The maximum length allowed for a sequence,
  defaults to 1000.\\
\item
  \texttt{maxGap}: The maximum gap between two events to be considered
  in a sequence, defaults to 1000.\\
\item
  \texttt{showID}: If sequence ID should be generated, defaults to
  FALSE.
\end{itemize}

An example can be the following:

\begin{Shaded}
\begin{Highlighting}[]
\NormalTok{frequentPatterns <-}\StringTok{ }\KeywordTok{runFrequentPatterns}\NormalTok{(}\DataTypeTok{algorithm =} \StringTok{"SPADE"}\NormalTok{, }
                                        \DataTypeTok{inputFile =} \StringTok{"FrequentPatternsExample.txt"}\NormalTok{, }
                                        \DataTypeTok{outputFile =} \StringTok{"FrequentPatternsExample_Results.txt"}\NormalTok{, }
                                        \DataTypeTok{minsup =} \FloatTok{0.5}\NormalTok{, }
                                        \DataTypeTok{showID =} \OtherTok{TRUE}\NormalTok{)}
\end{Highlighting}
\end{Shaded}

The function generates another text file saved as
``FrequentPatternsExample\_Results.txt'' in the current working
directory. It prints on the screen the number of mined sequences, the
time required, memory used and the location of where the output is
stored. It also, transforms the output to an R object and therefore
needs to be saved.

Additionally for FPM, \texttt{getIdDataFrame()} generates a data frame
object, indicating the presence or not of sequence for each patient id.
It is only applicable when \texttt{showID\ =\ TRUE} when running
\texttt{runFrequentPatterns()} and accepts as input the output
\textbf{.txt} file of the previous call.

\begin{Shaded}
\begin{Highlighting}[]
\KeywordTok{getIdDataFrame}\NormalTok{(}\DataTypeTok{inputFile =} \StringTok{"FrequentPatternsExample_Results.txt"}\NormalTok{)}
\end{Highlighting}
\end{Shaded}

\end{document}
